\documentclass{beamer}

\usepackage[slovene]{babel}
\usepackage{amsfonts,amssymb}
\usepackage[utf8]{inputenc}
\usepackage{lmodern}
\usepackage[T1]{fontenc}
\usepackage{graphicx}
\graphicspath{./project/media/videos/scene/1080p60/}
\usepackage{multimedia}

\usetheme{Warsaw}

\def\N{\mathbb{N}} % mnozica naravnih stevil
\def\Z{\mathbb{Z}} % mnozica celih stevil
\def\Q{\mathbb{Q}} % mnozica racionalnih stevil
\def\R{\mathbb{R}} % mnozica realnih stevil
\def\C{\mathbb{C}} % mnozica kompleksnih stevil
\def\H{\mathbb{H}} % mnozica kvaternionov
\def\E{\mathbb{E}} % evklidski prostor 
\def\Qe{\textbf{Q}_{e}} % mnozica versorjev
\def\Ue{\textbf{U}_{e}} % mnozica pravih versorjev
\def\1{\textbf{\emph{1}}}
\newcommand{\geslo}[2]{\noindent\textbf{#1} \quad \hangindent=1cm #2\\[-1pc]}
\newcommand{\dotpr}[2]{\langle #1, #2 \rangle}
\newcommand{\conj}[1]{\overline{#1}}


\def\qed{$\hfill\Box$}   % konec dokaza
\def\qedm{\qquad\Box}   % konec dokaza v matematičnem načinu
\newtheorem{izrek}{Izrek}
\newtheorem{trditev}{Trditev}
\newtheorem{posledica}{Posledica}
\newtheorem{lema}{Lema}
\newtheorem{opomba}{Opomba}
\newtheorem{definicija}{Definicija}
\newtheorem{zgled}{Zgled}
\newtheorem{primer}{Primer}
\newtheorem{zgledi}{Zgledi uporabe}
\newtheorem{zglediaf}{Zgledi aritmetičnih funkcij}
\newtheorem{oznaka}{Oznaka}

\title{Rotacije opisane s Kvaternioni}
\author{Timotej Mlakar}
\institute{Fakulteta za matematiko in fiziko \\
Oddelek za matematiko}
\date{\today}

\begin{document}


%%%%%%%%%%%%%%%%%%%%%%%%%%%%%%%%%%%%%%%%%%%%%%%%%%%%%%%%%%%%%%%%%%%%%

\begin{frame}
\titlepage
\end{frame}

%%%%%%%%%%%%%%%%%%%%%%%%%%%%%%%%%%%%%%%%%%%%%%%%%%%%%%%%%%%%%%%%%%%%%

\begin{frame}
   \begin{definicija}
      Naj bo $V$ 4-razsežen vektorski prostor z bazo $\{\1, i, j, k\}$.
      Elemente $V$ označimo $\textbf{q} = q_{0}\1 + q_{1}i + q_{2}j + q_{3}k = q_{0} + \vec{q}$.
      Vektorski prostor $V$ opremimo s operacijo množenja:
      \begin{center}
         $\1 \1 = \1, \hspace{1em}  \1i = i, \hspace{1em} \1j = j, \hspace{1em} \1k = k,$\\
         $ij = k, \hspace{1em} jk = i, \hspace{1em} ki = j,$\\
         $i^2 = j^2 = k^2 = ijk = -1\1$.
      \end{center}
      Tedaj $V$ postane 4-razsežna algebra nad $\R$, ki jo označimo s $\H$ in imenujemo \emph{Kvaternionska Algebra}.
   \end{definicija}

\end{frame}

%%%%%%%%%%%%%%%%%%%%%%%%%%%%%%%%%%%%%%%%%%%%%%%%%%%%%%%%%%%%%%%%%%%%%

\begin{frame}
   \begin{oznaka}
      \[
      \conj{q} = q_{0} - \vec{q}.
      \]
   \end{oznaka}

   \pause
   \begin{definicija}
      Naj bo $q \in \H$. Inverz $q$ za množenje je je tedaj 
      \[
      q^{-1} = \dfrac{1}{q\conj{q}} \conj{q}   
      \]
   \end{definicija}
\end{frame}

%%%%%%%%%%%%%%%%%%%%%%%%%%%%%%%%%%%%%%%%%%%%%%%%%%%%%%%%%%%%%%%%%%%%%

\begin{frame}
   \begin{definicija}
      Na $\H$ vpeljemo skalarni produkt, sicer za $p, q \in \H$:
      \[
      \dotpr{p}{q} = \frac{1}{2}\big( \conj{p}q + \conj{q}p \big)
      \]
      Norma, porojena s skalarnim produktom je
      \[
      ||q|| = |q| = \sqrt{\dotpr{q}{q}}   
      \]
   \end{definicija}

   \begin{opomba}
      Za vsak $q \in \H$ je $\dotpr{q}{q} = q\conj{q} = \conj{q}q$ in
      \[
        |q| = \sqrt{q\conj{q}} 
      \]
   \end{opomba}

   \pause
   \begin{opomba}
      Norma na $\H$ je multiplikativna.
   \end{opomba}
\end{frame}

%%%%%%%%%%%%%%%%%%%%%%%%%%%%%%%%%%%%%%%%%%%%%%%%%%%%%%%%%%%%%%%%%%%%%

\begin{frame}
   \begin{oznaka}
      \[
         \Qe = \left\{ q \in \H; |q| = 1 \right\}
      \]
      \[
        \Ue = \left\{ u \in \H; |u| = 1 \wedge u = \vec{u} \right\} 
      \]
   \end{oznaka}

   \pause
   \begin{opomba}
      Za $u \in \Ue$ velja
      \[
        u^2 = -1. 
      \]
      Za poljubna $u, v \in \Ue$ velja
      \[
      \dotpr{u}{v} = 0 \iff uv + vu = 0 \iff uv = -vu.
      \]
   \end{opomba}
\end{frame}

%%%%%%%%%%%%%%%%%%%%%%%%%%%%%%%%%%%%%%%%%%%%%%%%%%%%%%%%%%%%%%%%%%%%%

\begin{frame}
   \begin{trditev}
      Naj bo $q \in \Qe$. Obstajata $\theta \in \R$ in $u \in \Ue$, da je
      \[
        q  = \cos\theta + u\sin\theta.
      \]
   \end{trditev}

   \begin{opomba}
      Polarni zapis kvaterniona ni enoličen.
   \end{opomba}
\end{frame}

%%%%%%%%%%%%%%%%%%%%%%%%%%%%%%%%%%%%%%%%%%%%%%%%%%%%%%%%%%%%%%%%%%%%%

\begin{frame}
   \begin{trditev}
      Naj bosta $p, q \in \Qe$ taka, da $\exists u\in \Ue$, da je $p = e^{u\theta}$ in $q = e^{u\varphi}$,
      za neka $\theta, \varphi \in \R$.
      Tedaj je $pq = qp$.
   \end{trditev}
\end{frame}

%%%%%%%%%%%%%%%%%%%%%%%%%%%%%%%%%%%%%%%%%%%%%%%%%%%%%%%%%%%%%%%%%%%%%

\begin{frame}
   \begin{definicija}
      Naj bosta $p, q \in \Qe$. Definiramo preslikavo $C_{p,q} : \H \to \H$
      \[
      C_{p,q} x := pqx.   
      \]
   \end{definicija}

   \pause
   \begin{definicija}
      Posebej označimo preslikavo $C = C_{q, \conj{q}}$, za $q = e^{u\theta}; u \in \Ue, \theta \in \R$
      \[
      Cx := C_{q, \conj{q}}x = e^{u\theta}xe^{-u\theta}.   
      \]
   \end{definicija}
\end{frame}

%%%%%%%%%%%%%%%%%%%%%%%%%%%%%%%%%%%%%%%%%%%%%%%%%%%%%%%%%%%%%%%%%%%%%

\begin{frame}
   \begin{izrek}
      Naj bo $u \in \Ue$ in $\theta \in \R$ ter $q = e^{u\theta}$. Preslikava $C = C_{q, \conj{q}}$ je 
      rotacija ravnine, pravokotne na $u$ za kot $2\theta$.
   \end{izrek}
\end{frame}

%%%%%%%%%%%%%%%%%%%%%%%%%%%%%%%%%%%%%%%%%%%%%%%%%%%%%%%%%%%%%%%%%%%%%

\end{document}