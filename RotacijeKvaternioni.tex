\documentclass[a4paper,12pt]{article}

\usepackage[slovene]{babel}
\usepackage{amsfonts,amssymb,amsmath}
\usepackage[utf8]{inputenc}
\usepackage[T1]{fontenc}
\usepackage{lmodern}
\usepackage{graphicx}


\def\N{\mathbb{N}} % mnozica naravnih stevil
\def\Z{\mathbb{Z}} % mnozica celih stevil
\def\Q{\mathbb{Q}} % mnozica racionalnih stevil
\def\R{\mathbb{R}} % mnozica realnih stevil
\def\C{\mathbb{C}} % mnozica kompleksnih stevil
\def\H{\mathbb{H}} % mnozica kvaternionov
\def\E{\mathbb{E}} % evklidski prostor 
\def\Qe{\textbf{Q}_{e}} % mnozica versorjev
\def\Ue{\textbf{U}_{e}} % mnozica pravih versorjev
\def\1{\textbf{\emph{1}}}
\newcommand{\geslo}[2]{\noindent\textbf{#1} \quad \hangindent=1cm #2\\[-1pc]}
\newcommand{\dotpr}[2]{\langle #1, #2 \rangle}

\def\qed{$\hfill\Box$}   % konec dokaza
\def\qedm{\qquad\Box}   % konec dokaza v matematičnem načinu
\newtheorem{izrek}{Izrek}
\newtheorem{trditev}{Trditev}
\newtheorem{posledica}{Posledica}
\newtheorem{lema}{Lema}
\newtheorem{opomba}{Opomba}
\newtheorem{definicija}{Definicija}
\newtheorem{zgled}{Zgled}

\title{Rotacije opisane s kvaternioni \\ 
\Large Seminar}
\author{Timotej Mlakar \\
Fakulteta za matematiko in fiziko \\
Oddelek za matematiko}
\date{\today}

\begin{document}


%%%%%%%%%%%%%%%%%%%%%%%%%%%%%%%%%%%%%%%%%%%%%%%%%%%%%%%%%%%%%%%%%%%%%


\maketitle


%%%%%%%%%%%%%%%%%%%%%%%%%%%%%%%%%%%%%%%%%%%%%%%%%%%%%%%%%%%%%%%%%%%%%

\section{Uvod}

Rotacije $\R^3$ navadno opisujemo z linearnimi preslikavami oziroma njim pripadajočimi matrikami.
Zaradi narave matričnega množenja so lahko take operacije precej računsko časovno in prostorsko zahtevne. Tako lahko rotacije
$\R^3$ predstavimo kot stranske učinke transformacij $\E^4 \simeq \H$.


Najprej se spomnimo rotacij na $\R^2 \simeq \C$. Naj bo $w = \frac{v}{|v|}$ za poljuben $v \in \C$.
Preslikava $\varphi : \C \rightarrow \C : \varphi (z) = wz$ je bijektivna preslikava, ki zavrti celotno kompleksno ravnino za
kot $arg(z)$ okoli izhodišča.


Če $v$ zapišemo v polarnem zapisu kot $|z| e^{i \theta}$, je tedaj preslikava 
\begin{center}
   $\varphi : \left[ 0, 2\pi\right] \times \C \rightarrow \C :$\\
   \vspace*{1ex}
   $\varphi(\theta, z) = z e^{i\theta}$
\end{center}
zvezno odvedljiva na $\left[0, 2\pi\right] \times \C$. Za fiksen $z \in \C$ preslikava $\varphi$ opiše krožnico z radijem $|z|$,
za fiksen $\theta$ pa preslikava opiše rotacijo ravnine za kot $\theta$.

Vemo torej, da se vsak $z \in \C$ da zapisati v polarnih koordinatah. Spomnemo se zapisa
\begin{center}
   $z = |z|e^{i\theta} = |z|\text{cos}\theta + |z|i\text{sin}\theta$,
\end{center}
kjer je $\theta \in \R$. Zapis ni enoličen, saj nam vsaka $\theta' = \theta + 2k\pi; k \in \Z$ opiše isto kompleksno število.
Tak zapis bomo v podobnem smislu uporabili kasneje.

Definiramo $\varPhi : \R^2 \rightarrow \C : (x,y) \mapsto \text{x} + i\text{y}$. S preprostim računom pokažemo, da je $\varPhi$ izomorfizem.

Vidimo, da namesto množenja vektorja z matriko lahko rotacijo realne ravnine predstavimo s preprostim množenjem dveh kompleksnih števil.
To motivira podobni razmislek za rotacije v $\R^3$.
%%%%%%%%%%%%%%%%%%%%%%%%%%%%%%%%%%%%%%%%%%%%%%%%%%%%%%%%%%%%%%%%%%%%%

\section{Kvaternionska algebra}
\subsection{Definicije in oznake}

\begin{definicija}
Naj bo $V$ $4$-razsežen vektorski prostor nad $R$. Izberemo bazo $\left\{\1, i, j, k\right\}$. Elementi $V$ so oblike $\textbf{q} = q_{0}\1 + q_{1}i + q_{2}j + q_{3}k = q_{0} + \vec{q}.$
Vektorski prostor $V$ opremimo z operacijo množenja tako, da definiramo množenje njegovih baznih elementov, in sicer
\begin{center}
   $\1 \1 = \1, \hspace{1em}  \1i = i, \hspace{1em} \1j = j, \hspace{1em} \1k = k,$\\
   $ij = k, \hspace{1em} jk = i, \hspace{1em} ki = j,$\\
   $i^2 = j^2 = k^2 = ijk = -1\1$.
\end{center}
Naj bosta $p, q \in \H$. Definiramo seštevanje in množenje s skalarjem kot običajno
\begin{center}
   $p + q = \left( p_{0} + q_{0} \right) + \left( \vec{p} + \vec{q} \right)$, \\
   $ \lambda q = \lambda \left(q_{0} + \vec{q} \right) = \lambda q_{0} + \left(\lambda \vec{q}\right) $.
\end{center}
Prav tako definiramo običajno množenje v skladu z definicijo množenja baznih elementov. Tedaj lahko
produkt $pq$ napišemo kot
\begin{center}
   $pq = (p_{0} + q_{0} - \vec{p} \vec{q}) + (p_{0}\vec{q} + q_{0}\vec{p} + \vec{p}\times\vec{q})$,
\end{center}
kjer je $\vec{p}\vec{q}$ običajni skalarni produkt v $\R^3$.
Tedaj $V$ postane $4$-razsežna algebra nad $\R$. Označimo $\H$ in jo imenujemo \emph{Kvaternionska algebra}.
\end{definicija}

\begin{opomba}
Za $p,q \in \H, \lambda \in \R$ velja
\begin{center}
      $(\lambda p)q = p(\lambda q) = \lambda (pq)$.
\end{center}
\end{opomba}

% tukej pride mal zgodovine o kvaternionih, zgodba od Hamiltona

\begin{definicija}
Naj bo $q = q_{0} + \vec{q}\in \H$. S $\overline{q} = q_{0} -\vec{q}$ označimo \emph{konjugirani kvaternion} q.
\end{definicija}
Velja, da je $q\overline{q} \in \R$. Tako lahko definiramo še 
\begin{center}
   $q^{-1} = \dfrac{1}{q\overline{q}} \overline{q}$.
\end{center}
Prav tako lahko vidimo da je $\overline{p \cdot q} = \overline{q} \cdot \overline{p}$. Ker množenje kvaternionov ni komutativno,
v splošnem $\overline{pq} \neq \overline{qp}$. Ker je $\H$ algebra, je na njej smiselno definirati skalarni produkt.

\begin{definicija}
Naj bosta $p,q \in \H$. Definiramo skalarni produkt kvaternionov
\begin{center}
   $\langle p,q \rangle = \frac{1}{2} (\overline{p}q + \overline{q}p)$.
\end{center}
Norma porojena s skalarnim produktom je tedaj
\begin{center}
   $|q| = ||q|| = \sqrt{\langle q, q\rangle}$.
\end{center}
\end{definicija}

\begin{opomba}
Iz definicije norme takoj sledi $\langle q, q\rangle = q\overline{q} = \overline{q}q$.
Podobno kot absolutna vrednost na $\R$ in $\C$ je norma na kvaternionih multiplikativna.
\end{opomba}
Za poljubna $p,q \in \H$ torej velja $|pq| = |p||q|$. Oglejmo si $|pq|^2$
\begin{center}
   $|pq|^2 = \langle pq, pq \rangle = pq\overline{pq}$.
\end{center}
Spomnimo se, da $\overline{p \cdot q} = \overline{q} \cdot \overline{p}$. Torej je
\begin{center}
   $pq\overline{pq} = p\cdot q\cdot \overline{q} \cdot \overline{p} = p |q|^2 \overline{p}$.
\end{center}
Ker je $|q|^2$ skalar, pri množenju komutira s kvaternioni. Torej
\begin{center}
   $p|q|^2\overline{p} = |q|^2p\overline{p} = |q|^2 |p|^2 = |p|^2 |q|^2$.
\end{center}
Sledi torej $|pq| = |p||q|$.

Podobno kot pri rotaciji kompleksne ravnine, kjer množimo s števili iz enotske krožnice, tukaj potrebujemo 
enotske kvaternione.

\begin{definicija}
   Naj bo $q \in \H$. Kvaternion q imenujemo \emph{versor} oziroma \emph{enotski kvaternion}, če velja
   $|q| = 1$. Če je $q \in \H$ poljuben $|q| \neq 1$ versor kvaterniona q dobimo z normiranjem. Označimo ga z $U_{q} = \dfrac{q}{|q|}$. 
   Množico versorjev označimo s $\Qe$

   Če velja $u \in \H, u = \vec{u}$ in $|u| = 1$, kvaternion u imenujemo \emph{čisti} oziroma \emph{pravi} versor.
   Množico pravih versorjev označimo z $\Ue$.

   Kvaternione oblike $q = q_{0}\1, q_{0} \in \R$ imenujemo \emph{skalarni kvaternioni.}
\end{definicija}
%%%% TUKI SE NEKI MANKA SAM NEVEM KAJ TOCN
\begin{opomba}
   Za čista versorja $u, v \in \Ue$ velja da $\dotpr{u}{v} = 0 \iff uv + vu = 0$.
\end{opomba}
Naj bosta $u, v \in \Ue$. Za poljuben versor iz $\Ue$ velja $\overline{u} = -u$. Pogledamo $\dotpr{u}{v}$ :
\begin{center}
   $\dotpr{u}{v} = \frac{1}{2}(\overline{u}v + \overline{v}u) = \frac{1}{2}(-uv -vu) = -\frac{1}{2}(uv + vu)$.
\end{center}
Od tu sledi da $\dotpr{u}{v} = 0 \iff uv + vu = 0$.

Pomembna opazka tu je še naslednja: naj bo $u \in \Ue$. Ker $|u| = 1$ sledi, da je $u$ neničeln kvaternion.
Ker je $\Ue \subset \H$ in je $\H$ algebra, je vsak neničenli kvaternion obrnljiv. Vemo torej, da obstaja tak $u^{-1}$ da je
\begin{center}
   $uu^{-1} = 1$.
\end{center}
Ker je za poljuben $q \in \H, q^{-1} = \dfrac{1}{\overline{q}q} \overline{q}$, za $u \in \Ue$ pa velja $u\overline{u} = |u|^2$, je 
\begin{center}
   $u^{-1} = \dfrac{\overline{u}}{|u|^2} = \dfrac{-u}{1} = -u$.
\end{center}
Če združimo te dve dejstvi sledi, da
\begin{center}
   $uu^{-1} = -uu = -u^2 = 1 \Rightarrow u^2 = -1$.
\end{center}
\section{Eulerjeva funkcija}


\begin{definicija}
Za vse $n \in \N$ s $\varphi(n)$ označimo število 
celih števil iz množice $\{1, 2, \ldots, n\}$, ki so tuja številu $n$.
Preslikavo $\varphi: \N \to \N$ imenujemo \em{Eulerjeva funkcija}.
\end{definicija}

\begin{zgled}
Tabela \ref{fi} prikazuje izračun prvih šest vrednosti funkcije $\varphi(n)$. V $n$-ti vrstici so 
krepko natisnjena števila med $1$ in $n$, ki so tuja številu $n$. Slika~\ref{fi100} pa grafično prikazuje prvih 100 vrednosti funkcije $\varphi(n)$.
\begin{table}[h]
\[
\begin{array}{clc}
 n & \{1,2,\ldots, n\}          & \varphi(n)       \\
 \hline
 1 & \{{\bf 1}\}                    &     1      \\
 2 & \{{\bf 1},2 \}                &     1      \\
 3 & \{{\bf 1,2},3 \}             &     2      \\
 4 & \{{\bf 1},2,{\bf 3},4 \} &     2      \\
 5 & \{{\bf 1,2,3,4},5 \}       &     4      \\
 6 & \{{\bf 1},2,3,4,{\bf 5},6 \} &     2
\end{array}
\] 
\caption{Vrednosti funkcije $\varphi(n)$ za $n = 1,2,\ldots,6$}\label{fi}
\end{table}

\begin{figure}[h]
% \includegraphics{fi100.pdf}
\caption{Vrednosti funkcije $\varphi(n)$ za $n = 1,2,\ldots,100$}\label{fi100}
\end{figure}
\end{zgled}

Računanje $\varphi(n)$ po definiciji je pri velikem $n$ zelo zamudno. Vendar ima 
Eulerjeva funkcija lepe lastnosti, zaradi katerih lahko njeno vrednost izračunamo
tudi pri velikem argumentu, če ga le znamo razcepiti na prafaktorje.

Če je $p$ praštevilo, med števili $1,2,\ldots, p$ edinole število $p$ ni tuje številu $p$,
torej je $\varphi(p) = p-1$. Skoraj prav tako preprosto lahko poiščemo vrednost $\varphi(n)$,
če je $n$ potenca nekega praštevila.

\begin{trditev}
\label{fipp}
Naj bo $p$ praštevilo in $k \in \N$. Potem je $\varphi(p^k) = p^k-p^{k-1}$.
\end{trditev}

\noindent
{\em Dokaz:\/} Število $a$ je tuje številu $p^k$ natanko tedaj, ko ni večkratnik praštevila $p$.
Med števili $1,2,\ldots, p^k$ je natanko $p^k/p = p^{k-1}$ večkratnikov števila $p$, torej je $\varphi(p^k) = p^k-p^{k-1}$. 
\qed

\begin{izrek}
\label{fimult}
Eulerjeva funkcija je multiplikativna.
\end{izrek}

\noindent
{\em Dokaz:\/} Vzemimo tuji naravni števili $a$ in $b$. Zapišimo vsa števila med $1$ in $ab$
v obliki tabele z $a$ vrsticami in $b$ stolpci:
\[
\begin{array}{cccc}
1 & 2 & \cdots & \ b \\
b+1 & b+2 & \cdots & \ 2b \\
2b+1 & 2b+2 & \cdots & \ 3b \\
\vdots & \vdots & \cdots & \ \vdots \\
(a-1)b+1 & (a-1)b+2 & \cdots & \ ab 
\end{array}
\]
Za vsako število velja, da je tuje številu $ab$ natanko tedaj, ko je tuje številu $a$ in tuje številu $b$.
Vrednost $\varphi(ab)$ lahko torej dobimo tako, da preštejemo, koliko je v gornji tabeli števil, ki so tuja
tako številu $a$ kot tudi številu $b$. 

Števila v posameznem stolpcu dajejo vsa isti ostanek pri deljenju z $b$. Torej so bodisi vsa tuja številu
$b$ bodisi mu ni tuje nobeno od njih. Stolpcev, katerih elementi so tuji številu $b$, je toliko, kot je
takih števil v prvi vrstici tabele, teh pa je ravno $\varphi(b)$.

Različna števila v posameznem stolpcu dajo različne ostanke pri deljenju z $a$. Če namreč števili
$k_1 b + r$ in $k_2 b + r$, kjer je $0 \le k_1, k_2 \le a-1$, dasta isti ostanek pri deljenju z $a$, je njuna razlika $(k_1 - k_2) b$
deljiva z $a$. Ker sta števili $a$ in $b$ tuji, sledi, da je z $a$ deljiva razlika $k_1 - k_2$.
To pa je možno le, če je $k_1 = k_2$, saj je $-(a-1) \le k_1 - k_2 \le a-1$. Ker je dolžina stolpca
enaka $a$, dobimo pri deljenju elementov stolpca z $a$ ravno vse možne ostanke $0, 1,\ldots, a-1$.
Torej je v vsakem stolpcu $\varphi(a)$ števil tujih $a$. 

To velja tudi za $\varphi(b)$ stolpcev, katerih elementi so tuji številu $b$. Potemtakem je v gornji tabeli
$\varphi(b)\varphi(a)$ števil, ki so tuja tako številu $b$ kot tudi številu $a$. 
Torej je $\varphi(ab) = \varphi(a)\varphi(b)$, kar pomeni, da je Eulerjeva funkcija multiplikativna. \qed


\begin{zgled}
Izračunajmo $\varphi(10^k)$. Ker je $10^k = 2^k 5^k$, je po izreku \ref{fimult} in trditvi \ref{fipp}
\[
\varphi(10^k)\ =\ \varphi(2^k)\varphi(5^k)\ =\ (2^k - 2^{k-1})(5^k - 5^{k-1})\ =\ 4\times 10^{k-1}.
\]
\end{zgled}


\begin{posledica}
\[
\varphi(n)\ =\ n \times \prod_{p\,|\,n} \left(1 - \frac{1}{p}\right),
\]
\end{posledica}
kjer $p$ preteče vse različne prafaktorje števila $n$.

\noindent
{\em Dokaz:\/} Naj bo $n = \prod_{i=1}^r p_i^{k_i}$,
kjer so $p_1, p_2, \ldots, p_r$ različna praštevila in $k_1, k_2, \ldots, k_r \in \N$. Po izreku 
\ref{fimult} in trditvi \ref{fipp} je potem
\begin{eqnarray*}
\varphi(n) &=& \prod_{i=1}^r \varphi\left(p_i^{k_i}\right)
\ =\ \prod_{i=1}^r \left(p_i^{k_i} - p_i^{k_i-1}\right) \\
 &=& \left(\prod_{i=1}^r p_i^{k_i}\right) \times \prod_{i=1}^r \left(1 - \frac{1}{p_i}\right)
\ =\ n \times \prod_{p\,|\,n} \left(1 - \frac{1}{p}\right). \qedm
\end{eqnarray*}


\begin{trditev}
Za vse $n \in \N$ velja enačba
\begin{equation}
\label{fisum}
\sum_{d\,|\,n} \varphi(d)\ =\  n,
\end{equation}
kjer $d$ preteče vse pozitivne delitelje števila $n$.
\end{trditev}

\noindent
{\em Dokaz:\/} Za vse delitelje $d$ števila $n$ označimo
\[
A_d\ =\ \left\{\frac{k n}{d};\ k \in \Z,\ 0 \le k < d,\ D(k,d) = 1\right\}.
\]
Recimo, da je $k_1 n/d_1 = k_2 n/d_2$, kjer je $D(k_1,d_1) = D(k_2,d_2) = 1$. Potem 
je $k_1 d_2 = k_2 d_1$, od koder sledi, da $d_1$ deli $d_2$ in obratno, kar pomeni,
da je $d_1 = d_2$. Od tod zaključimo, da so si množice $A_d$ paroma tuje,
torej je
\[
\left|\bigcup_{d\,|\,n} A_d\right|\ =\ \sum_{d\,|\,n} |A_d|\ =\ \sum_{d\,|\,n} \varphi(d).
\]
Po drugi strani pa je
\[
\bigcup_{d\,|\,n} A_d\ =\ \{0, 1, \ldots, n-1\}.
\]
Res, naj bo $k n/d \in A_d$. Ker $d$ deli $n$, je število $k n/d$ celo, iz $0 \le k < d$ pa sledi $0 \le k n/d < n$,
torej $k n/d \in \{0, 1, \ldots, n-1\}$. Vzemimo zdaj še poljuben $j \in \{0, 1, \ldots, n-1\}$
in označimo: $k = j/D(n,j)$, $d = n/D(n,j)$. Potem je $j = k D(n,j) = k n/d \in A_d$.

To pa pomeni, da je $\left|\bigcup_{d\,|\,n} A_d\right| = n$ in izrek je dokazan. \qed

\begin{izrek}[Eulerjev izrek]
Naj bosta $n \in \N$ in $a \in \Z$ tuji števili. Potem je
\[
a^{\varphi(n)} \equiv 1 \pmod{n}.
\]
\end{izrek}

\noindent
{\em Dokaz:\/} Naj bodo $k_1, k_2, \ldots, k_{\varphi(n)}$ vsa števila med $1$ in $n$, ki so tuja $n$. Če za indeksa $i,j \in \{1,2,\ldots,\varphi(n)\}$ velja $k_i a \equiv k_j a \!\!\!\pmod{n}$, sledi $n | (k_i a - k_j a)$ in zato $n | (k_i - k_j)$, saj sta števili $n$ in $a$ tuji. To pa je mogoče le, če je $i = j$. Števila  $k_1 a, k_2 a, \ldots, k_{\varphi(n)} a$ so torej med seboj paroma nekongruentna po modulu $n$. Ker so tuja številu $n$, je množica njihovih ostankov pri deljenju z $n$ enaka množici $\{k_1, k_2, \ldots, k_{\varphi(n)}\}$. Zato je $k_1 a\cdot k_2 a\cdots k_{\varphi(n)} a \equiv k_1\cdot k_2 \cdots k_{\varphi(n)} \pmod{n}$, od tod pa po krajšanju s produktom $ k_1\cdot k_2 \cdots k_{\varphi(n)}$, ki je tuj številu $n$, dobimo $a^{\varphi(n)} \equiv 1 \!\!\!\pmod{n}$. \qed

\begin{posledica}[mali Fermatov izrek]
Naj bo $p$ praštevilo in $a \in \Z$ celo število, ki ni deljivo s $p$. Potem je
\[
a^{p-1} \equiv 1 \pmod{p}.
\]
\end{posledica}

%%%%%%%%%%%%%%%%%%%%%%%%%%%%%%%%%%%%%%%%%%%%%%%%%%%%%%%%%%%%%%%%%%%%%


\section{M\"obiusova funkcija}


\begin{definicija}
Za vse $n \in \N$ naj bo
\[
\mu(n)\ =\ \left\{
\begin{array}{cl}
0, & \mbox{če\ } n \mbox{\ deljiv s kvadratom praštevila,} \\
(-1)^r, & \mbox{sicer,}
\end{array}
\right.
\]
kjer je $r$ število različnih prafaktorjev števila $n$.
Preslikavo $\mu: \N \to \Z$ imenujemo \em{M\"obiusova funkcija}.
\end{definicija}

\begin{zgled}
Tabela \ref{mi} prikazuje prvih nekaj vrednosti funkcije $\mu(n)$. 
\begin{table}[h]
\[
\begin{array}{c|*{10}{r}}
   n   & 1 & 2 & 3 & 4 & 5 & 6 & 7 & 8 & 9 & 10 \\
\hline
\mu(n) & \ \ 1 & -1 & -1 & \ \ 0 & -1 & \ \ 1 & -1 & \ \ 0 & \ \ 0 & \ \ 1
\end{array}
\]
\caption{Vrednosti funkcije $\mu(n)$}\label{mi}
\end{table}
\end{zgled}


\begin{izrek}
\label{mimult}
M\"obiusova funkcija je multiplikativna.
\end{izrek}

\noindent
{\em Dokaz:\/} Vzemimo tuji naravni števili $a$ in $b$. Če je število $ab$ deljivo s kvadratom praštevila,
velja to tudi za $a$ ali za $b$. V tem primeru je torej $\mu(ab) = 0 = \mu(a)\mu(b)$.
Če pa število $ab$ ni deljivo s kvadratom praštevila, velja to tudi za $a$ in za $b$. Naj bo $r$ število različnih
prafaktorjev števila $a$, $s$ pa število različnih prafaktorjev števila $b$. Potem je število različnih 
prafaktorjev števila $ab$ enako $r+s$, torej je v tem primeru $\mu(ab) = (-1)^{r+s} = (-1)^r (-1)^s = \mu(a)\mu(b)$.
\qed


\begin{trditev}
Za vse $n \in \N$ velja enačba
\begin{equation}
\label{mirek}
\sum_{d\,|\,n} \mu(d)\ =\ \left\{
\begin{array}{ll}
1, & n = 1, \\
0, & n > 1,
\end{array}
\right.
\end{equation}
kjer $d$ preteče vse pozitivne delitelje števila $n$.
\end{trditev}

\noindent
{\em Dokaz:\/} 
Zadošča seštevati po tistih deliteljih $d$ števila $n$, ki imajo same različne prafaktorje (sicer je
$\mu(d) = 0$). Imenujmo takšne delitelje {\em enostavni}. Naj bo $r$ število različnih prafaktorjev
števila $n$. Število enostavnih deliteljev števila $n$, ki imajo natanko $k$ prafaktorjev, je potem ${r \choose k}$,
prispevek takega delitelja h gornji vsoti pa znaša $\mu(d) = (-1)^k$. Torej je
\[
\sum_{d\,|\,n} \mu(d)\ =\ 
\sum_{k=0}^r (-1)^k {r \choose k} \ =\ 
\left\{
\begin{array}{ll}
1, & r = 0, \\
0, & r > 0
\end{array}
\right.\ =\ \left\{
\begin{array}{ll}
1, & n = 1, \\
0, & n > 1.
\end{array}
\right. \qedm
\]

\begin{opomba}
Enačbo (\ref{mirek}) bi lahko uporabili tudi za (rekurzivno) definicijo funkcije $\mu(n)$:
\[
\mu(n)\ =\ \left\{
\begin{array}{cl}
1, & n = 1, \\
-\displaystyle\sum_{d\,|\,n, \,d < n} \mu(d), & n > 1.
\end{array}
\right.
\]
\end{opomba}

%%%%%%%%%%%%%%%%%%%%%%%%%%%%%%%%%%%%%%%%%%%%%%%%%%%%%%%%%%%%%%%%%%%%%
M\"obiusova funkcija igra pomembno vlogo pri {\em M\"obiusovem obratu}, ki nam omogoča
izraziti aritmetično funkcijo $f(n)$, če poznamo funkcijo $g(n) = \sum_{d\,|\,n} f(d)$,
kjer $d$ preteče vse pozitivne delitelje števila $n$.

\begin{izrek} \em{(M\"obiusov obrat)} \ 
Za aritmetični funkciji $f, g$ velja:
\[
g(n)\ =\ \sum_{d\,|\,n} f(d)\quad \Longleftrightarrow\quad f(n)\ =\ \sum_{d\,|\,n} \mu\left(\frac{n}{d}\right)g(d)
\]
\end{izrek}

\noindent
{\em Dokaz:\/} 
Najprej vzemimo, da je $g(n) = \sum_{d\,|\,n} f(d)$ za vse $n \in \N$. Potem je
\begin{eqnarray*}
\sum_{d\,|\,n} \mu\left(\frac{n}{d}\right)g(d)
 &=& \sum_{d\,|\,n} \mu\left(\frac{n}{d}\right)\sum_{k\,|\,d} f(k)
\ =\ \sum_{k\,|\,n} f(k) \sum_{k\,|\,d\,|\,n} \mu\left(\frac{n}{d}\right) \\
 &=& \sum_{k\,|\,n} f(k) \sum_{a\,|\,(n/k)} \mu\left(a\right)\ =\ f(n).
 \end{eqnarray*}
Drugo enakost smo dobili z zamenjavo vrstnega reda seštevanja, tretjo z uvedbo nove spremenljivke $a = n/d$,
četrta pa sledi iz (\ref{mirek}).

Vzemimo zdaj, da je $f(n)\ =\ \sum_{d\,|\,n} \mu\left(\frac{n}{d}\right)g(d)$ za vse $n \in \N$. Potem je
\begin{eqnarray*}
\sum_{d\,|\,n} f(d)
 &=& \sum_{d\,|\,n} \sum_{k\,|\,d} \mu\left(\frac{d}{k}\right) g(k)
\ =\ \sum_{k\,|\,n} g(k) \sum_{k\,|\,d\,|\,n} \mu\left(\frac{d}{k}\right) \\
 &=& \sum_{k\,|\,n} g(k) \sum_{b\,|\,(n/k)} \mu\left(b\right)\ =\ g(n).
 \end{eqnarray*}
Drugo enakost smo dobili z zamenjavo vrstnega reda seštevanja, tretjo z uvedbo nove spremenljivke $b = d/k$,
četrta pa sledi iz (\ref{mirek}). \qed


\begin{zgled}
\label{tau}
\begin{itemize}
\item Iz enačbe (\ref{fisum}) sledi z M\"obiusovim obratom, da je 
\[
\varphi(n)\ =\ \sum_{d\,|\,n}\mu\left(\frac{n}{d}\right)d.
\]
\item Za vse $n \in \N$ s $\tau(n)$ označimo število vseh pozitivnih deliteljev števila $n$.
Torej je $\tau(n) = \sum_{d\,|\,n} 1$, od koder sledi z M\"obiusovim obratom, da je 
\[
\sum_{d\,|\,n}\mu\left(\frac{n}{d}\right)\tau(d)\ =\ 1.
\]
\item Za vse $n \in \N$ s $\sigma(n)$ označimo vsoto vseh pozitivnih deliteljev števila $n$.
Torej je $\sigma(n) = \sum_{d\,|\,n} d$, od koder sledi z M\"obiusovim obratom, da je 
\[
\sum_{d\,|\,n}\mu\left(\frac{n}{d}\right)\sigma(d)\ =\ n.
\]
\end{itemize}
\end{zgled}






%%%%%%%%%%%%%%%%%%%%%%%%%%%%%%%%%%%%%%%%%%%%%%%%%%%%%%%%%%%%%%%%%%%%%


\section{Kolobar aritmetičnih funkcij}

\begin{definicija}
Za aritmetični funkciji $f, g: \N \to \C$ in za vse $n \in \N$ naj bo 
\[
(f * g)(n) \ =\ \sum_{d\,|\,n} f(d)g\left(\frac{n}{d}\right).
\]
Aritmetična funkcija $f * g$ je {\em Dirichletova konvolucija\/} funkcij $f$ in $g$.
\end{definicija}

\begin{trditev}
\label{kolo}
Naj bodo $f$, $g$, $h$ aritmetične funkcije. Potem velja:
\begin{itemize}
\item[\rm (i)] $f * g \ =\ g * f$,
\item[\rm (ii)] $(f * g) * h \ =\ f * (g * h)$,
\item[\rm (iii)] $f * (g + h) \ =\ f * g + f * h$.
\end{itemize}
\end{trditev}

\noindent
{\em Dokaz:\/} 
\begin{itemize}
\item[\rm (i)] 
Trditev sledi iz zapisa Dirichletove konvolucije v simetrični obliki
\begin{equation}
\label{dk}
(f * g)(n) \ =\ \sum_{d e = n} f(d)g(e),
\end{equation}
kjer seštevamo po vseh urejenih parih naravnih števil $(d,e)$, katerih produkt je enak $n$.

\item[\rm (ii)] 
Z uporabo enačbe (\ref{dk}) izračunamo
\begin{eqnarray*}
((f * g) * h)(n) &=& \sum_{d e = n} (f * g)(d)h(e)
\ =\ \sum_{d e = n} \left(\sum_{a b = d} f(a)g(b)\right)h(e) \\
 &=& \sum_{a b e = n} f(a)g(b)h(e)
 \ =\ \sum_{a c = n} f(a) \sum_{b e = c} g(b)h(e) \\
 &=& \sum_{a c = n} f(a) (g * h)(c)
 \ =\ (f * (g * h))(n).
\end{eqnarray*}
Četrto enakost smo dobili z uvedbo nove spremenljivke $c = b e$.


\item[\rm (iii)]
Z uporabo enačbe (\ref{dk}) izračunamo
\begin{eqnarray*}
(f * (g + h))(n) &=& \sum_{d e = n} f(d)(g+h)(e)
\ =\ \sum_{d e = n} f(d)(g(e)+h(e)) \\
 &=& \sum_{d e = n} f(d)g(e) + \sum_{d e = n} f(d)h(e) \\
 &=& (f * g + f * h)(n). \qedm
\end{eqnarray*}

\end{itemize}

Iz trditve \ref{kolo} sledi, da je množica vseh aritmetičnih funkcij $f: \N \to \C$ z operacijama $+$ in $*$ komutativen kolobar. Imenujemo ga {\em Dirichletov kolobar} in označimo z $\cal D$. 

Funkcija $\varepsilon \in {\cal D}$, ki za vse $n \in \N$ zadošča enačbi
\[
\varepsilon(n)\ =\ \left\{
\begin{array}{ll}
1, & n = 1, \\
0, & n > 1,
\end{array}
\right.
\]
je enica kolobarja $\cal D$, saj za vse $f \in {\cal D}$ in $n \in \N$ velja
\[
(f * \varepsilon)(n) \ =\ \sum_{d e = n} f(d)\varepsilon(e) \ =\ f(n)\varepsilon(1) \ =\ f(n).
\]
Brez težav se lahko prepričamo tudi, da je $\cal D$ cel kolobar in da je funkcija $f \in {\cal D}$
obrnljiva natanko tedaj, ko $f(1) \ne 0$.

Zdaj lahko enačbo (\ref{mirek}) prepišemo v obliki
\[
\mu * \mathbf{1}\ = \ \varepsilon,
\]
kjer $\mathbf{1}$ označuje konstantno funkcijo z vrednostjo $1$. Z drugimi besedami, M\"obiusova funkcija
je inverz konstantne funkcije $\mathbf{1}$ glede na Dirichletovo konvolucijo:
\[
\mu \ =\  \mathbf{1}^{-1}.
\]
M\"obiusov obrat lahko torej zapišemo v obliki
\[
g\ =\ f * \mathbf{1} \quad \Longleftrightarrow \quad f\ =\ g * \mu,
\]
kjer njegova veljavnost postane očitna. Zgled \ref{tau} pa lahko prepišemo v obliki
\begin{eqnarray*}
\varphi * \mathbf{1} \ =\ {\rm id}_\N &\Longrightarrow& \varphi\ =\ \mu * {\rm id}_\N, \\
\tau \ =\ \mathbf{1} * \mathbf{1} &\Longrightarrow& \mu * \tau \ =\ \mathbf{1}, \\
\sigma \ =\ {\rm id}_\N * \mathbf{1} &\Longrightarrow& \mu * \sigma \ =\ {\rm id}_\N.
\end{eqnarray*}


\section*{Angleško-slovenski slovar strokovnih izrazov}

\geslo{proper}{pravi}

\geslo{pure}{pravi, čisti}

\geslo{versor}{versor, enotski kvaternion}

\geslo{dot product}{skalarni produkt}

\geslo{by-product}{stranski učinek}




\begin{thebibliography}{1}
\bibitem{AiZ}
M.~Aigner in G.~M.~Ziegler, \emph{Proofs from THE BOOK}, 2.\ izdaja, Springer, Berlin--Heidelberg--New York, 2001.
\bibitem{CaW}
N.~Calkin in H.~S.~Wilf, Recounting the rationals,
\emph{Amer.~Math.~Monthly}  \textbf{107}  (2000),  360--363.
\bibitem{Gra}
J.~Grasselli, \emph{Elementarna teorija števil}, DMFA -- založništvo, Ljubljana, 2009.
\end{thebibliography}





\end{document}